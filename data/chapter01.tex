\chapter{绪论}
\label{chap:introduction}
\section{研究背景与意义}

假如你是你是一名临床医学与人工智能交叉领域的科研人员,
你致力于构建一个基于患者身体指标信息的抑郁症诊断模型,从而实现抑郁症的早期精准识别。
模型可以通过年龄、睡眠数据、激素水平、基因数据等信息来挖掘潜在的抑郁症特征。
然而,为了提高诊断的准确性,不可避免地需要收集患者的敏感信息来生成数据集。
与此同时,随着学术交流合作的日渐频繁,其它研究者可能会请求获取数据集来完成分析、验证假设。
这带来了严峻的隐私保护挑战,作为信息的收集者,你有义务保证患者的敏感信息。
如何衡量隐私的泄露情况,应该采用什么样的方法保护隐私成为了重要课题。

隐去部分隐私信息是一个看起来有效的方法。例如在公开数据集时,可以将姓名、生日、电话号码等标识信息隐藏。
然而,这样的隐私保护方法具有局限性。攻击者可以利用一些辅助数据和推理方法,将隐去标识信息的数据重新定位到个体。
假设攻击者拥有将姓名与基因对应的辅助数据集,那么通过比对基因信息,就可以将数据对应到个体。
这种攻击被称为关联分析攻击,已经有研究表明,相关的攻击案例并不少见。\cite{narayanan2016precautionary}

差分隐私是一种具备严格数学证明的隐私保护模型,为解决上述问题提供了思路。
它定量衡量了算法的隐私保护程度,并通过添加精心设计的噪声来确保单个数据不会显著影响输出,
从而在保证算法的可用性的同时,实现对个体隐私的保护。

差分隐私与传统密码学都致力于保护隐私信息,但两者关注的方向有所不同,
后者注重防止输出以外的隐私泄露,而前者假设输出本身就会包含隐私信息,
希望通过设计更好的信息发布形式减少隐私泄露。

在一个$n$个点$m$条边的加权无向图$G=(V,E)$中,割是顶点的一个二划分$(X,V\backslash X)$,
其权重为跨越这个划分的边权之和。给定一对顶点$s,t\in V$,$s-t$最小割是满足$s\in X,t\in V\backslash X$
的权重最小的割$(X,V\backslash X)$,也就是说,它是将$s$与$t$分隔开的权重最小的割。
$s-t$最小割问题与$s-t$最大流问题对偶,即根据最大流最小割定理,$s-t$最小割值等于$s-t$最大流值。\cite{ford1956maximal}
相类似的,全局最小割问题是求图中权值最小的割,它反映了图的连通程度,这同样是图论领域的一个基本问题。

差分隐私定量的衡量了隐私的保护程度,因此对算法的有了额外的要求,
即在几乎相同的两个输入下,算法的计算过程也应当几乎相同。
以最小割算法为例,这意味着在两个输入仅相差一条边的情况下,要求输出的最小割结果
的概率增幅不能超过一个极小的常数系数。设计差分隐私下的最小割算法,有助于推广最小割算法的更多实际应用,
也能加深对差分隐私下算法设计方法的探索。
差分隐私下的最小割算法的设计难点在于,算法需要在保持差分隐私性的同时,
控制噪声带来的误差,保证输出输出的可用性。

\section{研究现状}

在过去的几十年中,人们提出了众多算法来解决最小割问题。

1993年,Karger等人提出了一种基于删边的求解最小割的$O(n^4\log n)$的随机算法,
他们的工作同时说明了不同的最小割的数量不超过$\frac{n(n-1)}2$。\cite{karger1993global}
该算法简洁清晰,他们证明了在随机选择边收缩的情况下,
指定的最小割有不可忽视的概率在算法结束时得到保留,因此重复足够多次算法的执行,
就可以找到一个最小割。1996年,Karger等人改进了算法,将多次独立的重复执行合并成树的若干分支,
从而提高了效率,得到了一个求解最小割的$O(n^2\log^3 n)$的随机算法。\cite{karger1996new}
值得注意的是,Karger等人的收缩算法能以高概率找到所有的最小割。

2000年,Karger提出了一种基于树包装的求解最小割的$ O(m\log^3n)$的随机算法。\cite{karger2000minimum}
这个算法也可以解决找到所有最小割的变体问题,复杂度为$O(n^2\log n)$。
树包装是一个生成树的集合,满足图上的每条边被生成树包含的权重和不超过其边权。
他们还称割与生成树为$k$关联,当且仅当割的边集与生成树边集的并集大小不超过$k$。
树包装可以生成一个大小为$O(\log n)$的生成树集合,满足每个最小割都与其中至少$\frac13$的生成树$2$关联。
如此一来,只要找到这组生成树$2$关联的所有割并判断其割值,就能算出所有的最小割。
Karger的树包装算法是目前最好的求解最小割的随机算法。

2021年,Li提出了一种针对Karger算法去随机化的$O(m^{1+o(1)})$的确定性算法。\cite{li2021deterministic}
Li的去随机化算法是目前最好的求解最小割的确定性算法。

1976年,Dinitz等人设计了一种叫作仙人掌图表示法的数据结构,以一个稀疏化图来表示所有的最小割。\cite{dinitz1976structure}\cite{fleiner2009quick}
前面提到的几个最小割算法虽然也能完成所有最小割的计算,但由于直接存储数量为$O(n^2)$的最小割复杂度较高,
因此找到的最小割以中间结果的形式存储在算法中,扩展能力有限。
Dinitz等人提出的仙人掌图表示法做到了用一个规模为$O(n)$的图表示所有最小割。
具体来说,他们为图$G$建立了一个仙人掌图$\Gamma$和映射$\varphi:V_G\rightarrow V_\Gamma$,
且对于任意$G$中的最小割$(X,V\backslash X)$,
其对应到$\Gamma$中的点集$\varphi(X)$与$\varphi(V\backslash X)$,都满足存在一个$\Gamma$中的最小割将其分隔开。
Dinitz等人也通过仙人掌图表示法,证明了图最小割的数量不超过$\frac{n(n-1)}2$,这也是该结论最早的证明。

2009年,Karger基于其树包装最小割算法,提出了一个构造仙人掌图表示法的$O(m\log^4n)$的随机算法。\cite{karger2009near}
该算法的思想是借助树包装算法计算了所有点与边的极小最小割,再通过这部分信息设计一个递归过程完成仙人掌图表示法的构造。
2024年,He等人将仙人掌图表示法构建算法进行优化,得到了一个$O(m\log^3n)$的随机算法,
此外,他们还完成了算法的去随机化,得到了一个$O(m\text{polylog}(n))$的确定性算法。\cite{he2024cactus}

差分隐私下的最小割算法也在近年来有所研究。
2010年,Gupta等人提出了一种基于拉普拉斯机制的差分隐私最小割算法。\cite{gupta2010differentially}
他们的指数算法实现了$\varepsilon$-差分隐私,
并将得到的近似最小割与真实最小割的割值误差控制在$O(\ln n/\varepsilon)$。
此外他们还提出了$(\varepsilon,\delta)$-差分隐私的多项式时间复杂度算法,
作为差分隐私算法的高效选择。

近几年,同样是稀疏化图的Gomory-Hu树在隐私化上取得了一定进展。
Gomory-Hu树以树的形式保有了全点对的$s-t$最小割值,
具体来说,它保证$s-t$最小割的值等于Gomory-Hu树上$s$与$t$之间路径的边权最小值。
2021年,Li等人提出了一个构建$(1+\epsilon)$-近似Gomory-Hu树
的$\tilde O(m+n^{3/2}\epsilon^{-2})$的随机算法。\cite{li2021approximate}
这一工作基于他们此前提出的最小隔离割方法得出。\cite{li2020deterministic}
2024年,Aamand等人对算法进行了隐私化,
得到了一个加性误差为$\tilde O(m/\varepsilon)$的构建Gomory-Hu树的$\varepsilon$-差分隐私随机算法。\cite{li2021approximate}

2024年,Liu等人提出了一个针对图的隐私化算法,
该算法能以$(\varepsilon,\delta)$-差分隐私地发布处理过后的图,并保证图上最小割的误差为$\tilde O(\frac{\sqrt {nm}}{\epsilon})$。\cite{liu2024almost}

\section{本文的主要内容}

本文的目标是设计一个能够输出多个近似最小割的差分隐私算法。

TODO