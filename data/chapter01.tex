\chapter{绪论}
\label{chap:introduction}
\section{研究背景与意义}

随着人工智能的高速发展,数据的重要性愈发凸显,在部分领域的研究中,
常涉及隐私数据的使用。以临床医学与人工智能交叉研究为例,
若科研人员希望构建基于患者身体指标的抑郁症诊断模型,实现抑郁症早期精准识别。
则模型可能需要诸如年龄、睡眠质量、激素水平、基因序列等数据。
因此,为提升模型诊断能力,收集患者的敏感信息难以避免。
此外,随着学术交流合作的不断深入,其它研究者可能申请获取数据用于分析验证,
这使得隐私保护面临巨大挑战。数据收集方有责任保障患者的信息安全。
所以,如何量化隐私泄露的风险、选择有效的隐私保护方法,已成为亟待解决的重要课题。

隐去隐私标识信息是一种常见的隐私保护手段。例如,在公开数据集时,
通常会对姓名、生日、电话号码等可识别个人身份的信息进行隐藏处理。
然而,这种保护方法存在固有缺陷。攻击者可借助辅助数据集并结合推理分析技术,
重新建立匿名数据与具体个体之间的映射关系。
例如,若攻击者持有包含姓名与基因对应关系的辅助数据集,就能通过基因信息比对,
实现对目标个体的精确识别。
上述攻击方式被称为关联分析攻击,已经有研究表明,
此类攻击在实际场景中屡见不鲜。\cite{narayanan2016precautionary}

差分隐私作为一种基于严格数学证明的隐私保护模型,对应对上述挑战提供了有效解决方案。
该模型通过量化算法的隐私保护程度,来要求算法添加精心设计的噪声,
以确保单个数据对输出结果的影响不显著,
从而在保证算法可用性的同时,实现对个体隐私的可靠保护。\cite{vadhan2017complexity}

差分隐私与传统密码学均以隐私保护为目标,但两者侧重点存在差异。
传统密码学聚焦于防范输出结果以外的隐私泄露风险,
而差分隐私则是基于输出内容本身包含隐私信息的假设,
通过优化信息发布机制来降低隐私泄露概率。

在一个包含$n$个点、$m$条边的加权无向图$G=(V,E)$中,割指顶点的二划分$(X,V\backslash X)$,
其权重定义为跨越该划分的边权总和。
对于给定顶点对$s,t\in V$,$s-t$最小割是满足$s\in X,t\in V\backslash X$
条件下的权重最小的割$(X,V\backslash X)$,即实现$s$与$t$分离的最小权重割集。
根据最大流最小割定理,
$s-t$最小割问题与$s-t$最大流问题存在对偶关系,
两者在数值上相等。\cite{ford1956maximal}
类似地,全局最小割问题旨在求解图中权值最小的割集,全局最小割的割值能够有效衡量图的连通性,
是图论研究中的经典基础问题。

差分隐私通过量化指标精确衡量隐私的保护程度,
这对算法设计提出了新的约束:
当两个输入近乎相同时,算法的计算过程也应当保持高度相似。
以最小割算法为例,若两个输入图仅存在一条边的差异,其输出的最小割结果
的概率变化需被限制在极接近$1$的常系数范围内。

设计满足差分隐私的最小割算法,不仅能够拓展算法在实际场景中的应用范围,
还有助于加深对差分隐私框架下算法设计方法论的研究。
此类算法的核心设计难点在于,如何在严格遵循差分隐私限制的同时,
有效控制因添加噪声引入的误差,并同时确保算法输出结果的可用性。

\section{研究现状}

在过去的几十年间,人们提出了众多算法来解决最小割问题。

1993年,Karger等人提出了一种基于边收缩的随机算法,用于求解最小割问题,其时间复杂度为$O(n^4\log n)$。
该研究同时证明,图中不同的最小割的数量上限为$\frac{n(n-1)}2$。\cite{karger1993global}
该算法构造简洁,易于理解。算法证明了在随机选择边收缩时,
指定最小割有一定概率在算法终止时得以保留,通过重复执行算法,
即可以高概率找到一个最小割。1996年,Karger等人对算法进行了改进,
通过将多次独立重复执行的过程整合为树的分支结构,提升了算法效率,
得到时间复杂度$O(n^2\log^3 n)$的最小割求解随机算法。\cite{karger1996new}
此外,由于每个最小割的计算在该收缩算法中是同时进行的,因此算法在以高概率找到一个最小割的同时,
也能以高概率找到所有的最小割。

2000年,Karger提出了一种基于树包装的随机算法,同样是用于求解最小割问题,其时间复杂度为$ O(m\log^3n)$。\cite{karger2000minimum}
这个算法同样适用于求解所有的最小割,且解决这一变体问题的时间复杂度为$O(n^2\log n)$。
树包装是一个生成树的集合,其中图的每条边被各生成树包含的权重总和不超过其自身边权。
Karger等人定义了割与生成树$k$关联,当且仅当割的边集与生成树边集的并集大小不超过$k$。
通过树包装,可以构建一个规模为$O(\log n)$的生成树集合,
使得每个最小割都至少与集合中$\frac13$的生成树存在$2$关联。
基于这个特性,通过枚举与这些生成树$2$关联的所有割,并计算其割值,即可获取全部最小割。
目前,Karger的树包装算法仍是求解最小割问题的最优随机算法。

2021年,Li提出了一种针对Karger算法去随机化的确定性算法,其时间复杂度为$O(m^{1+o(1)})$。\cite{li2021deterministic}
该算法是目前求解最小割问题的最优确定性算法。

1976年,Dinitz等人提出仙人掌图表示法(cactus representation),
这个数据结构以一个稀疏化图的形式表示了所有的最小割。\cite{dinitz1976structure}\cite{fleiner2009quick}
前文提到的最小割算法虽也能计算所有最小割,但直接存储规模为$O(n^2)$的最小割集代价过高,
因此最小割仅以中间结果的形式暂时存储,导致算法可扩展性受限。
而仙人掌图表示法创新性地用一个规模为$O(n)$的图实现全部最小割的表示。
具体而言,仙人掌图表示法由为图$G$建立的仙人掌图$\Gamma$和映射$\varphi:V_G\rightarrow V_\Gamma$构成;
给定的仙人掌图和映射满足,
任意$G$中的最小割$(X,V\backslash X)$
对应的$\Gamma$中的点集$\varphi(X)$与$\varphi(V\backslash X)$
一定可被至少一个$\Gamma$中的最小割分隔。
Dinitz等人也通过仙人掌图表示法,证明了图最小割的数量不超过$\frac{n(n-1)}2$,这也是该结论最早的证明。

2009年,Karger基于树包装最小割算法,提出了一个构造仙人掌图表示法的随机算法,
时间复杂度为$O(m\log^4n)$。\cite{karger2009near}
该算法首先固定一根节点,并计算所有点与边的极小最小割,
然后算法通过点的次极小最小割生成一棵树,最后通过边的极小最小割对树进行连边,形成仙人掌图,完成构造。
2024年,He等人将仙人掌图表示法构建算法进行优化,得到了时间复杂度为$O(m\log^3n)$的随机算法,
同时,通过算法去随机化处理,进一步得到了时间复杂度$O(m\text{polylog}(n))$的确定性算法。\cite{he2024cactus}

近年来,差分隐私下的最小割算法研究取得进展。
2010年,Gupta等人提出一种基于拉普拉斯机制的差分隐私最小割算法。\cite{gupta2010differentially}
该算法实现了$\varepsilon$-差分隐私,
其近似最小割与真实最小割的割值误差界为$O(\ln n/\varepsilon)$。
此外,他们还设计出满足$(\varepsilon,\delta)$-差分隐私的多项式时间复杂度算法,
为输出一个最小割的差分隐私算法提供了高效解决方案。

Gomory-Hu树是一种与仙人掌图表示相似的重要结构,
近年来其结构性质及构造算法的隐私化研究取得重大突破。
Gomory-Hu树用树存储了全点对的$s-t$最小割值,
具体来说,图中$s-t$最小割值等于Gomory-Hu树上$s$与$t$之间路径边权的最小值。
2021年,Li等人提出了一个时间复杂度为$\tilde O(m+n^{3/2}\epsilon^{-2})$的随机算法,
用以构建$(1+\epsilon)$-近似Gomory-Hu树。\cite{li2021approximate}
该算法基于其先前提出的最小隔离割方法。\cite{li2020deterministic}
2024年,Aamand等人对算法进行了隐私化改造,
得到了一个构建Gomory-Hu树的$\varepsilon$-差分隐私的随机算法,
树表示的最小割与真实值相比的加性误差为$\tilde O(m/\varepsilon)$。\cite{li2021approximate}

2024年,Liu等人提出了一个面向图的隐私化算法,
该算法能以$(\varepsilon,\delta)$-差分隐私地发布一个合成图,
并保证合成图上最小割的值与其在原图中的真实割值的加性误差为$\tilde O(\frac{\sqrt {nm}}{\epsilon})$。\cite{liu2024optimal}

\section{研究内容与组织架构}

差分隐私的概念从提出至今已有二十年左右的发展历程,其中差分隐私图算法在近几年被广泛关注与研究。
由于最小割不唯一,因此最小割问题有两个计算目标:求一个最小割和求所有最小割构成的割集。
无论是1993年的Karger收缩算法还是2021年的Li确定性算法都能对这两个计算目标进行求解。
然而,2010年Gupta等人提出的差分隐私最小割算法仅能输出一个最小割。

所以,本文重点研究图结构与其所有最小割的性质关联,从仙人掌表示、最小割数量的敏感度出发,
结合拉普拉斯机制,指数机制,$k$优选择机制这三个差分隐私算法和Karger最小割求解算法,
来进行新算法的设计。


围绕上述内容,本文共分为六章,具体组织如下:
\begin{itemize}
    \item 第一章阐明研究的背景与意义,综述领域内的研究现状,提出核心问题与创新点。
    \item 第二章给出图论与差分隐私的形式化表示体系,回顾重要算法和定理。
    \item 第三章分析了仙人掌图表示的结构与$p$割、$t$割的关联,
    并提出了同一个图$G$对应的仙人掌图表示非唯一性问题,最后通过构造算法定义了标准形式,
    并给出了一个高效的仙人掌图表示标准化算法。
    \item 第四章分析了最小割数量的敏感度,并基于仙人掌图表示建立了精细的敏感度分析框架,并完成敏感度上界的分析。
    \item 第五章渐进地提出了三个求解最小割的算法,第一个算法基于隐私化图算法和割值筛选法,
    第二个算法基于差分隐私最小割数量和隐私化$k$优选择机制,
    第三个算法在第二个算法的基础上,用指数机制和Karger算法进行了优化,实现了较低加法误差下的差分隐私近似最小割求解。
    \item 第六章总结了本文的主要创新,并对未来的优化方向进行了设想。
\end{itemize}

