\chapter{绪论}
\label{chap:introduction}
\section{研究背景与意义}

随着人工智能的高速发展,数据的重要性愈发凸显,部分领域的研究
涉及隐私数据的使用。以临床医学与人工智能交叉研究为例,
若科研人员希望构建基于患者身体指标的抑郁症诊断模型,实现抑郁症早期精准识别。
则模型可能需要诸如年龄、睡眠质量、激素水平、基因序列等数据。
因此,为提升模型诊断能力,收集患者的敏感信息难以避免。
此外,随着学术交流合作的不断深入,其它研究者可能申请获取数据用于分析验证,
这使得隐私保护面临巨大挑战。数据收集方有责任保障患者的信息安全。
所以,如何量化隐私泄露的风险、选择有效的隐私保护方法,已成为亟待解决的重要课题。

隐去隐私标识信息是一种常见的隐私保护手段。例如,在公开数据集时,
通常会对姓名、生日、电话号码等可识别个人身份的信息进行隐藏处理。
然而,这种保护方法存在固有缺陷。攻击者可借助辅助数据集并结合推理分析技术,
重新建立匿名数据与具体个体之间的映射关系。
在上文的例子中,若攻击者持有包含姓名与基因对应关系的辅助数据集,就能通过基因信息比对,
实现对目标个体的精确识别。
上述攻击方式被称为身份识别攻击,已经有研究表明,
此类攻击在实际场景中屡见不鲜。\cite{narayanan2016precautionary}

作为应对,数据的处理和计算可以由数据拥有方(或其它受信任的第三方)来完成。
具体来说,在数据请求者给出询问内容后,
数据拥有方会对询问进行计算,最后返回结果。
在这种模式下,数据集没有被直接公开,
因此对隐私的保护进一步增强。
然而,对单个数据的询问仍然可能导致数据集泄露,
多个非对单的查询进行组合后也可以差分地得到个体的信息。\cite{dinur2003revealing}
因此,数据拥有方应当为询问设立一套标准,
来控制返回结果中的隐私泄露程度。

差分隐私作为一种基于严格数学证明的隐私保护模型,对应对上述挑战提供了有效解决方案。
该模型通过量化询问中算法的隐私保护程度,来要求算法添加精心设计的噪声,
以确保单个数据对输出结果的影响不显著,
从而在保证算法可用性的同时,实现对个体隐私的可靠保护。\cite{vadhan2017complexity}

差分隐私与传统密码学均以隐私保护为目标,但两者侧的重点存在差异。
传统密码学聚焦于防范输出结果以外的隐私泄露风险,
而差分隐私则是基于输出内容本身包含隐私信息的假设,
通过优化信息发布机制来降低隐私泄露概率。

本文聚焦于差分隐私下的最小割问题。
在一个包含$n$个点、$m$条边的加权无向图$G=(V,E)$中,割指顶点的二划分$(X,V\backslash X)$,
其权重定义为跨越该划分的边权总和。
常见的最小割问题有$s-t$最小割问题和全局最小割问题两种。
对于给定顶点对$s,t\in V$,$s-t$最小割是满足$s\in X,t\in V\backslash X$
条件下的权重最小的割$(X,V\backslash X)$,即实现$s$与$t$分离的最小权重割。
根据最大流最小割定理,
$s-t$最小割问题与$s-t$最大流问题存在对偶关系,
两者在数值上相等。\cite{ford1956maximal}
类似地,全局最小割问题旨在找到图的最小权重的割,
不难说明,全局最小割的权重等于所有点对$s.t$的$s-t$最小割权重的最小值。
最小割的权重能够衡量图的连通性,
因此如何求解最小割是图论领域的一个经典问题。

由于差分隐私为算法提出了新的要求,
即须控制衡量隐私保护程度的参数,
因此经典的最小割求解方法在不再适用。
差分隐私要求算法当得到两个近乎相同的输入时,
应当有高概率相似的计算过程和结果。
对应到最小割算法当中,若两个输入图仅存在一条边的差异,
则其输出的每一种最小割的概率之比都接近$1$。

设计满足差分隐私的最小割算法,不仅能够拓展算法在实际场景中的应用范围,
还有助于加深对差分隐私框架下算法设计方法论的研究。
此类算法的核心设计难点在于,如何在严格遵循差分隐私限制的同时,
有效控制因添加噪声引入的误差,并同时确保算法输出结果的可用性。

\section{研究现状}

在过去的几十年间,人们提出了众多算法来解决最小割问题。

1993年,Karger等人提出了一种基于边收缩的随机算法,用于求解最小割问题,其时间复杂度为$O(n^4\log n)$。
该研究同时证明,图中不同的最小割的数量上限为$\frac{n(n-1)}2$。\cite{karger1993global}
该算法构造简洁,易于理解。算法证明了在随机选择边收缩时,
指定最小割有一定概率在算法终止时得以保留,通过重复执行算法,
即可以高概率找到一个最小割。1996年,Karger等人对算法进行了改进,
通过将多次独立重复执行的过程整合为树的分支结构,提升了算法效率,
得到时间复杂度$O(n^2\log^3 n)$的最小割求解随机算法。\cite{karger1996new}
此外,每个最小割被找到的概率都可以通过上述方法说明,
因此算法在以高概率找到至少一个最小割的同时,
也以高概率能找到所有的最小割。

2000年,Karger提出了一种基于树包装的随机算法,同样是用于求解最小割问题,其时间复杂度为$ O(m\log^3n)$。\cite{karger2000minimum}
当算法用于解决寻找所有最小割这一变体问题时,
时间复杂度为$O(n^2\log n)$。
树包装是一个生成树的集合,其中图的每条边被各生成树包含的权重总和不超过其自身边权。
Karger等人定义割与生成树$k$关联为,割的边集与生成树边集的并集大小不超过$k$。
通过树包装,他们设计了一个构建规模为$O(\log n)$的生成树集合的算法,
且满足每个最小割都至少与集合中$\frac13$的生成树$2$关联。
在此基础上,枚举与这些生成树$2$关联的所有割,并计算其割值,即可获取全部最小割。
目前,Karger的树包装算法仍是求解最小割问题的最优随机算法。

2021年,Li提出了一种求解最小割问题的确定性算法,
其时间复杂度为$O(m^{1+o(1)})$。\cite{li2021deterministic}
这一工作的思路是将Karger的树包装算法去随机化。
目前,该算法是求解最小割问题的最优确定性算法。

1976年,Dinitz等人提出仙人掌图表示法(cactus representation),
这个数据结构以一个稀疏化图的形式表示了所有的最小割。\cite{dinitz1976structure}\cite{fleiner2009quick}
最小割的规模为$O(n^2)$,因此直接存储的代价较高。
前文提到的最小割算法能找到所有最小割,
但这些最小割存储在算法的过程变量中,
若要提取和利用需要额外开销,
因此扩展性受限。
仙人掌图表示法创新性地用一个规模为$O(n)$的图实现全部最小割的表示,
解决了存储开销问题并为面向图中所有最小割的算法提供了新思路。
对于给定图$G$,其仙人掌图表示法由仙人掌图$\Gamma$和映射$\varphi:V_G\rightarrow V_\Gamma$构成;
给定的仙人掌图和映射满足,
任意$G$中的最小割$(X,V\backslash X)$
对应的$\Gamma$中的点集$\varphi(X)$与$\varphi(V\backslash X)$
一定可被至少一个$\Gamma$中的最小割分隔。
Dinitz等人通过仙人掌图表示法证明了图最小割的数量不超过$\frac{n(n-1)}2$,这是该结论最早的证明。

2009年,Karger基于树包装最小割算法,提出了一个构造仙人掌图表示法的随机算法,
时间复杂度为$O(m\log^4n)$。\cite{karger2009near}
该算法首先固定一根节点,并计算所有点与边的极小最小割,
然后通过点的次极小最小割生成一棵树,最后通过边的极小最小割对树进行连边,形成仙人掌图,完成构造。
2024年,He等人将仙人掌图表示法构建算法进行优化,得到了时间复杂度为$O(m\log^3n)$的随机算法,
同时,通过算法去随机化处理,进一步得到了时间复杂度$O(m\text{polylog}(n))$的确定性算法。\cite{he2024cactus}

近年来,差分隐私下的最小割算法研究取得进展。
2010年,Gupta等人提出一种基于拉普拉斯机制的差分隐私最小割算法。\cite{gupta2010differentially}
该算法以$\varepsilon$-差分隐私得到原图的一个近似最小割,
且该近似最小割与真实最小割的割值误差界为$O(\ln n/\varepsilon)$。
此外,他们还设计出满足$(\varepsilon,\delta)$-差分隐私的多项式时间复杂度算法,
为差分隐私下的最小割问题提供了高效解决方案。

Gomory-Hu树是一种与仙人掌图表示相似的结构图,
其结构性质及构造算法的隐私化研究在近年来取得重大突破。
Gomory-Hu树以树的形式存储了全点对的$s-t$最小割值,
原图的点与Gomory-Hu树上的点一一对应,
且原图中$s-t$最小割值等于Gomory-Hu树中$s$与$t$之间路径上边权的最小值。
2021年,Li等人提出了一个时间复杂度为$\tilde O(m+n^{3/2}\epsilon^{-2})$的随机算法,
用以构建$(1+\epsilon)$-近似Gomory-Hu树。\cite{li2021approximate}
该算法基于其先前提出的最小隔离割方法。\cite{li2020deterministic}
2024年,Aamand等人对算法进行了隐私化改造,
得到了一个构建Gomory-Hu树的$\varepsilon$-差分隐私的随机算法,
且得到的最小割近似值与真实值相比的加性误差为$\tilde O(m/\varepsilon)$。\cite{li2021approximate}
Gomory-Hu树隐私化方法为本文提供了灵感,
即在最小割问题上,从一些特殊的结构出发完成隐私化。

2024年,Liu等人提出了一个图的隐私化算法,
该算法能以$(\varepsilon,\delta)$-差分隐私地发布一个合成图,
并保证合成图上割的值与其在原图中的真实割值的加性误差为$\tilde O(\frac{\sqrt {nm}}{\epsilon})$。\cite{liu2024optimal}
最小割是一类特殊的割,
因此该算法也为差分隐私下的最小割问题提供了一个求解路径。

\section{研究内容与组织架构}

差分隐私的概念从提出至今已有二十年左右的发展历程,其中差分隐私图算法在近几年被广泛关注与研究。
由于最小割不唯一,因此最小割问题有两个计算目标:
其一是找到至少一个最小割,
其二是找到所有最小割构成的割的集合。
Karger的边收缩算法、树包装算法以及2021年Li的确定性算法都能对这两个计算目标进行求解。
而在差分隐私算法方面,
现有的2010年Gupta等人提出的算法仅能输出单个最小割,
尚未有算法能够找出所有最小割,存在研究空白。

本文基于已有研究成果,探索用于寻找所有最小割的算法设计。
通过分析仙人掌表示及最小割数量的敏感度,
综合运用拉普拉斯机制、指数机制、$k$优选择机制这三种差分隐私算法,
结合图论中Karger最小割求解算法,完成了算法的设计。


围绕上述内容,本文共分为六章,具体组织如下:
\begin{itemize}
    \item 第一章阐明研究的背景与意义,综述领域内的研究现状,提出核心问题与创新点。
    \item 第二章给出图论与差分隐私的形式化表示体系,回顾重要算法和定理。
    \item 第三章分析了仙人掌图表示的结构与$p$割、$t$割的关联,
    并提出了同一个图$G$对应的仙人掌图表示非唯一性问题,最后通过构造算法定义了标准形式,
    并给出了一个高效的仙人掌图表示标准化算法。
    \item 第四章分析了最小割数量的敏感度,并基于仙人掌图表示建立了精细的敏感度分析框架,并完成敏感度上界的分析。
    \item 第五章渐进地提出了三个求解最小割的算法,第一个算法基于隐私化图算法和割值筛选法,
    第二个算法基于差分隐私最小割数量和隐私化$k$优选择机制,
    第三个算法在第二个算法的基础上,用指数机制和Karger算法进行了优化,实现了较低加法误差下的差分隐私近似最小割求解。
    \item 第六章总结了本文的主要创新,并对未来的优化方向进行了设想。
\end{itemize}

