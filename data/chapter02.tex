\chapter{符号表示与理论基础}

\section{图与最小割}

我们用$G=(V,E)$来表示一个$n$个点$m$条边无向图,其中点集是$V$,边集是$E$。
图中连接点$u$和点$v$的边用$(u,v)$表示。
若无向图带权,则我们用$w$来表示边权,也就是说,对于一条边$(u,v)\in E$,其边权为$w(u,v)$。
对于图上中的两个点集$V_1,V_2$,记连接两个点集的边集为
\begin{equation*}
    E(V_1,V_2)=\{(v_1,v_2)\in E|v_1\in V_1,v_2\in V_2\}
\end{equation*}
特别的,当$V_2=V\backslash V_1$时,$E(V_1,V_2)$可以简化为$E(V_1)$;
记连接两个点集的边权和为
\begin{equation*}
    w(V_1,V_2)=\sum_{v_1\in V_1,v_2,\in V_2}w(v_1,v_2)
\end{equation*}
特别的,当$V_2=V\backslash V_1$时,$w(V_1,V_2)$可以简化为$w(V_1)$。
在本文中,如无特殊说明,图允许重边的存在,但不允许自环边的存在,图是有限图,且保证图是连通图。

接下来,我们给出图的最小割及相关概念的定义。
\begin{definition}
   给定图$G=(V,E)$,一个割$R=(V_1,V_2)$是将顶点集合分成两个不相交的非空子集$V_1$和$V_2$,
   即满足$V_1\neq \emptyset,V_2\neq \emptyset,V_1\cap V_2=\emptyset,V_1\cup V_2=V$。
\end{definition}
割定义中的点集没有先后顺序,即$(V_1,V_2)=(V_2,V_1)$,
所以只需要给出一个点集$V$的非空真子集$V_1$就可以唯一确定一个割。
也就是说,割可以简化表示为$\Delta(V_1)=R(V_1,V\backslash V_1)$。
这个简化表示满足$\Delta(V_1)=\Delta(V_2)$当且仅当$V_1=V_2$或$V_1=V\backslash V_2$。
    \begin{definition}
    给定图$G=(V,E)$和图上的一个割$R=(V_1,V_2)$,
    割的边集为$E(V_1,V_2)$。
\end{definition}
与割的简化表示相对应,割$\Delta(V_1)$的边集也可以简化表示为$E(V_1)$。
\begin{definition}
    给定图$G=(V,E)$和图上的一个割$R=(V_1,V_2)$,
    割的容量为$ w(V_1,V_2)$。
\end{definition}
割的容量也叫作割值。只需要将割的边集中的边权进行求和,就可以得到割的容量。
因此,割的容量的另一个等价形式为$w(V_1)$。
\begin{definition}[点的度数]
    给定图$G=(V,E)$,点$v$的度数为
    \begin{equation*}
        \deg(v)=|\{e\in E|v\in e\}|
    \end{equation*}
\end{definition}
\begin{definition}
    给定图$G=(V,E)$,一个最小割$R=(V_1,V_2)$满足$R$是图$G$所有可能的割中容量最小的割。
\end{definition}
    我们用$R^*_{G}$表示图$G$的最小割集,$r^*_{G}\in R^*_{G}$表示一个最小割,
    $\Phi_G=w(r^*_{G})$表示图$G$的最小割的割值。
    此外,我们用$M_G=|R^*_{G}|$表示图$G$的最小割数量。
\begin{definition}
    给定图$G=(V,E)$,$\alpha$乘法近似,$\beta$加法近似最小割$R$满足$w(R)\leq \alpha ·\Phi_G+\beta$。
\end{definition}
在描述近似最小割时,若$\alpha=1$,则无需考虑该乘法参数,若$\beta=0$,则无需考虑该加法参数。
\begin{definition}[$S-T$最小割]
    给定图$G=(V,E)$和图上互不相交的两个非空点集$S,T\subset V$,
    $S-T$最小割是满足$S\subseteq V_1,T\subseteq V_2$的割$(V_1,V_2)$中权值最小的割。
\end{definition}
我们记$S-T$最小割的割值为$\Phi(S,T)$。当$S$和$T$均为只包含一个点的集合时,可以得到$s-t$最小割的定义。

\section{最小割数量的估计}

最小割并不唯一。例如,当图为一条边权均相同的链时,每一条边都对应一个最小割。
Karger给出了一个随机化的求解最小割的高效算法,他也通过这个算法对最小割的数量进行了估计。

\begin{definition}
    给定图$G=(V,E)$和点集的一个子集$X\subseteq V$,点收缩收缩过程为,在图$G$中新建一个点$x$,
    对于点$y\in V\backslash X$,其向$x$连一条边权为$\sum_{x'\in X}w(y,x')$的边(若边权为$0$则不连边),
    并将$X$及与其相连的边全部删除。
\end{definition}

\begin{definition}
    给定图$G=(V,E)$和图上的一条边$(u,v)\in E$,边$(u,v)$的收缩定义为对$\{u,v\}$这一点集执行点收缩。
\end{definition}

Karger算法的思路如下:以均匀分布来随机选择图的一条边,并对这条边进行边收缩,
重复该步骤直到图中的顶点数量等于一个预先设定的参数$k$为止。算法执行完时,如果一个割的割边集中没有边被收缩,那么我们说这个割是有效的。

Karger提出了下面的定理来估计最小割数量。
\begin{theorem}\cite{karger1993global}
    在算法进行到图被收缩至$k$个顶点时,一个给定的最小割有效的概率是$\Omega((n/k)^{-2})$。
\end{theorem}
当$k=2$时,给定的最小割仍然有效的概率为$\Omega((n/2)^{-2})$,且在$k$的这个取值下,有且仅有一个割有效,
因此,我们可以得到一个对最小割数量的估计。
\begin{theorem}
    给定图$G=(V,E)$,图中最小割数量至多为$n^2$。
\end{theorem}
Karger还将其算法进行了推广,由此可以得到一个对近似最小割数量的估计。
\begin{theorem}\cite{karger1994random}
    给定图$G=(V,E)$,图中$\alpha$乘法近似最小割的数量至多为$n^{2\alpha}$。
\end{theorem}

\section{仙人掌图表示法}

仙人掌图表示法是最早由Dinitz等人提出的结构图,该结构图保留了原图所有的最小割信息,且结构图是仙人掌图。

\begin{definition}
    图$G$为仙人掌图当且仅当,对于任意边$e\in V_G$都满足$e$至多属于一个简单环。
\end{definition}

\begin{theorem}[仙人掌图表示法]\cite{dinitz1976structure}
\label{cactus}
    给定带权图$G$,存在一个仙人掌图$\Gamma$和映射$\varphi:V_G\rightarrow V_\Gamma$,满足:
    \begin{itemize}
        \item 对于点$v_1,v_2\in V_G$,$\varphi(v_1)=\varphi(v_2)$当且仅当图$G$不存在最小割$R=(V_1,V_2)$使得$v_1\in V_1,v_2\in V_2$;
        \item 对于图$G$的任意一个最小割$R=(V_1,V_2)$,都满足$(\varphi(V_1),V_\Gamma\backslash\varphi(V_1))$是图$\Gamma$的一个最小割。
    \end{itemize}
\end{theorem}




\section{差分隐私}

差分隐私是一种针对敏感输入数据集计算的隐私定义,它聚焦于对个体隐私的保护。
通俗来说,差分隐私要求在两个几乎相同的输入数据下,算法的计算过程应当同样保持几乎一致。
当输入数据仅改变一个个体或者说一个元素时,任何输出结果的概率增幅不能超过一个很小的常数$e^\varepsilon$。
图论算法中,输入的元素单位为边,而边权可以视作叠加边的数量,
因此,图的差分隐私算法需要考察两个仅相差一条边的图的输出情况。
接下来,我们给出差分隐私在图论中的形式化定义。

\begin{definition}[边相邻]
    称图$G,G'$边相邻当且仅当满足以下条件:
    \begin{itemize}
        \item 顶点集相等:$V_G=V_{G'}$;
        \item 存在唯一边$(u,v)\in V^2$,使得$|w(u,v)-w_{G'}(u,v)|=1$;
        \item 对于任意其余边$(u',v')\in V^2\backslash \{(u,v)\}$,满足$w(u,v)=w_{G'}(u,v)$。
    \end{itemize}
\end{definition}

\begin{definition}[差分隐私]\cite{dwork2006differential}
    图算法$A$是$(\varepsilon,\delta)$差分隐私的,当且仅当对于任意的边相邻的图输入$G,G'$和输出值域的子集$O$,
    有
    \begin{equation*}
        \mathbb P[A(G)\in O]\leq e^\varepsilon\mathbb P[A(G')\in O]+\delta
    \end{equation*}
    特别地,如果$\delta=0$,算法满足$\varepsilon$差分隐私。
\end{definition}
当$\delta=0$时,差分隐私也被称为纯差分隐私。当$\delta\neq 0$时,差分隐私也被称为近似差分隐私。
近似差分隐私不能严格限定概率增幅,但是当$\delta$设定为一个极小的值时,仍然是一个有效的结果。

\begin{theorem}[基本组合]\cite{dwork2006calibrating}\cite{dwork2009differential}
    设$\varepsilon_1,\ldots,\varepsilon_t > 0$
    且$\delta_1,\ldots,\delta_t > 0$。
    若运行$t$个算法,其中第$i$个算法是$(\varepsilon_i,\delta_i)$差分隐私的,
    那么整个算法是$(\varepsilon_1 + \ldots + \varepsilon_t,\delta_1 + \ldots + \delta_t)$
    差分隐私的。
\end{theorem}

基本组合定理表明,一个差分隐私序列仍然具备差分隐私性。
这使得在设计算法时,可以将目标拆解成若干个差分隐私步骤,
从而降低了设计难度。

\begin{definition}[拉普拉斯分布]
    如果随机变量的概率密度函数分布为
    \begin{equation*}
        f(x|\mu,b)=\frac 1{2b} exp\left(-\frac{|x-\mu|}b\right)
    \end{equation*}
    那么它就是拉普拉斯分布。
\end{definition}

拉普拉斯分布的函数关于$x=\mu$轴对称,且对称轴的两侧分别是一个指数分布。拉普拉斯分布的期望为$\mu$,方差为$2b^2$。
特别的,当$\mu=0$时,我们记该分布为$\text{Lap}(b)$。

\begin{theorem}[拉普拉斯机制]\cite{dwork2014algorithmic}
    给定一个将图$G$映射到$\mathbb{R}^d$的函数$f$,
    其满足对于任意两个边相邻图$G$和$G'$,有$\|f(G)-f(G')\|_1 \leq \Delta$,
    则发布加入独立同分布随机噪声$X_i \sim \text{Lap} (\Delta/\varepsilon)$的结果
    \begin{equation*}
        f(G)+(X_1,\ldots,X_d)
    \end{equation*}
    满足$\varepsilon$差分隐私。
\end{theorem}

\begin{theorem}[差分隐私图]\cite{liu2024almost}
    令$\varepsilon \in \left(\frac{1}{n}, \frac{1}{2}\right) $和 
    $ 0 < \delta < \frac{1}{2} $为差分隐私参数。
    对于任意有$n$个点和$m$条边($m\geq n$)的带权图$G$,存在一个$(\varepsilon,\delta)$-差分隐私算法,
    能以至少$ 1 - o(1) $的概率输出一个合成图,
    满足对于合成图中任意不想交的点集$ S, T \subseteq V_G $,
    满足
    \begin{equation*}
        |w_G(S, T) - w_{\hat{G}}(S, T)| = O\left(\frac{\sqrt{nm}}{\varepsilon} \log^3 \left(\frac{n}{\delta}\right)\right)   
    \end{equation*}
    
\end{theorem}

拉普拉斯机制给出了一种差分隐私的方法,同时说明了敏感度$\Delta$与误差之间的关联。
定理中噪声拉普拉斯函数的系数表明,
敏感度越大,噪声的标准差也随之线性变大。

TODO:指数机制

TODO:top-k选择
