\eabstract{

\par This paper design a differentially private (DP) algorithm for computing 
the approximate minimum cuts of a weighted graph.
The research first analyzes the relationship between 
cactus representation and minimum cuts, identifies the
non-uniqueness issue, 
and innovatively designs a standardization algorithm.
This algorithm employs techniques such as separating $p$-cuts and $t$-cuts,
compressing redundant nodes, thereby
ensuring the uniqueness of the output while maintaining high computational efficiency.
Additionally, the paper explores the sensitivity of the minimum cut count and rigorously 
proves its upper bound. This finding provides 
a theoretical basis for the laplace mechanism in subsequent
differential privacy algorithms.

\par Based on the above results, this paper introduces three differentially
private algorithms for approximating the minimum cuts of weighted graphs:
The first algorithm privatizes the graph
and finds approximate minimum cuts in the resulting synthetic graph;
The second algorithm differentially privately releases the minimum cut value and
the minimum cut count, and identifies approximate minimum cuts through
top-$k$ selection mechanism; Building on the second algorithm, the third one
combines the exponential mechanism with Karger's contraction algorithm. 
These three algorithms are $(\varepsilon,\delta)$-DP, and
achieve an optimal additive error of $O(\frac{n\log n}{\varepsilon})$. 
This study strikes a balance between privacy protection
and accuracy.
Future work could further analyse the sensitivity of minimum cut count,
and explore a more efficient machanism to reduce the additive error.
}{Differential Privacy, Minimum Cut Problem, Cactus Representation}