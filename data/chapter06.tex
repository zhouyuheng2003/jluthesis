\chapter{总结与展望}
\section{工作总结}
最小割问题不仅在理论计算机领域有着重要的学术研究价值,同时在通信网络、芯片电路、系统设计、生物信息等领域也有着广泛的应用价值。
差分隐私使算法在更多数据敏感的场景得到应用成为了可能。本论文的主要工作为:
\begin{itemize}
    \item 定义了标准仙人掌图表示法,并给出了一个高效的标准化个仙人掌图表示法的算法。
    \item 定量分析了最小割数量的敏感度。
    \item 给出了一个加性误差为$O(\frac{n\log n}\varepsilon)$的差分隐私近似最小割算法。
\end{itemize}

\section{研究展望}
未来工作可从以下方向展开:
\begin{itemize}
    \item 在理解仙人掌图表示法的基础上分析多边形图表示法,以分析近似最小割本身的性质。
    \item 设计能差分隐私的输出仙人掌图表示的算法。
    \item 对本文中的算法进行进一步改进,以获取更优的加性误差结果。
\end{itemize}