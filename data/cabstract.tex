\cabstract{
    \par 本文聚焦差分隐私框架下的近似最小割问题展开系统性研究,提出一种高效的算法。
    研究以仙人掌图表示法为切入点,首先剖析仙人掌结构与最小割的内在联系,进而指出表示法的非唯一性问题,
    并创新性地设计标准化处理算法。该算法通过分离$p$割与$t$割、压缩冗余节点等策略,
    在确保了输出结果唯一性的同时,维持算法的高效性。
    此外,本文深入研究边相邻图定义下最小割数量的敏感度,
    严格证明了其上界,
    为后续差分隐私算法的噪声添加机制提供理论依据。

    \par 基于上述理论成果,本文提出了三种满足$(\varepsilon,\delta)$-差分隐私的近似最小割计算算法:
    1.对原图进行隐私化处理,然后用差分隐私发布的最小割值筛选出近似最小割;
    2.差分隐私地发布最小割的割值与数量,利用$k$优选择机制获取近似最小割;
    3.基于算法2,融合指数机制与Karger收缩算法,显著降低加性误差。
    其中,最优算法以$O(\frac{n\log n}{\varepsilon})$的加性误差输出
    具有数量保证的近似最小割,实现了隐私保护和结果可用性的平衡。
    本研究推进了差分隐私与图算法的交叉融合,
    为平衡效率、可用性与隐私保护的算法设计提供了新的技术路径。
    未来可进一步研究最小割数量的特性,探索更高效的隐私保护策略以降低算法加性误差。
}{差分隐私, 最小割问题, 仙人掌图表示法}
